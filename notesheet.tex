\documentclass[10pt]{article}
\usepackage{multicol}
\paperwidth=8.5in
\paperheight=11in
\begin{document}
\begin{multicols}{2}
Emily Dunham\\
ECE375 Midterm 1 Notesheet

"Store Accumulator Indirect with Pre-Decrement"

STA -(x) ; M(x) <- M(x)-1, M(M(x)) <- AC,

on the class cpu but with temporary register TEMP. An instruction consists of
16 bits: A 4-bit operation code and a 12-bit address. All operands are 16
bits. PC and MAR each contain 12 bits. AC, MDR, and TEMP each contain 16 bits,
and IR is 4 bits. Give the sequence of microoperations required to implement
the Execute cycles for the above STA -(x) instruction. Your solution should
result in minimum number of microoperations. Assume PC is currently pointing 
to the STA instruction and only PC and AC have the capability to 
increment/decrement itself. 

\begin{verbatim}
Fetch Cycle
Step 1: MAR <- PC;
Step 2: MDR <- M(MAR), PC <- PC+1
Step 3: IR <- MDRopcode, MAR<- MDRaddress 
; Read inst. & increment PC
Execute Cycle
Step 1: MDR <- M(MAR) 
; Get EA+1 from memory (i.e., M(x))
Step 2: TEMP <- AC ; Save AC to TEMP
Step 3: AC <- MDR ; Decrement EA+1
Step 4: AC <- AC -1
Step 5: MDR <- AC ; Store EA into M(x)
Step 6: M(MAR) <- MDR
Step 7: AC <- TEMP ; Restore AC
Step 8: MAR <- MDR ; Have MAR point to EA
Step 9: MDR <- AC 
; Store content of AC into M(M(x))=M(EA)
Step 10: M(MAR) <- MDR
; Note that it is also possible to perform 
; steps 1-2 and 6-7 at the same time.
\end{verbatim}


Consider the internal structure of a pseudo-CPU discussed in class. Suppose
the pseudo-CPU can be used to implement the hypothetical instructions given
below (Actually, these are instructions from PDP-8, which is the first
commercially successful minicomputer from the 60s). Give the sequence of
microoperations required to implement the Fetch and Execute cycle for each of
the instructions shown below. Your solution should result in minimum number of
microoperations. Assume PC is currently pointing to the instruction and only
PC and AC have the capability to increment itself.

(a) Logical AND: AND Y ; AC <- AC + M(Y)

(b) Increment and skip if zero: 
ISZ Y ; M(Y) <- M(Y) + 1, If(M(Y)+1=0) Then PC <- PC +1

(c) Deposit and clear the accumulator: DCA
Y ; M(Y) <- AC, AC <- 0

(d) Jump to subroutine: JMS Y ; M(Y) <- PC, PC <- Y + 1

Fetch Cycle

Step 1: MAR <- PC

Step 2: MDR <- M(MAR), PC <- PC+1

Step 3: IR <- MDRopcode, MAR <- MDRaddress

Step 4: Goto Execute cycle

AND Y ; AC <- AC + M(Y)

Execute Cycle

Step 1: MDR <- M(MAR)

Step 2: AC <- AC + MDR

Step 3: Goto Fetch cycle




\end{multicols}
\end{document}
