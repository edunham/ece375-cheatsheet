\documentclass[8pt]{article}
\usepackage{multicol}
\usepackage[margin=0.5in]{geometry}
\paperwidth=8.5in
\paperheight=11in
\begin{document}
\begin{multicols}{3}
Emily Dunham\\
ECE375 Midterm 1 Notesheet

\line(1,0){150}

Four dimensions of instruction sets:                                            
                                                                                
* Operations provided in the instruction set.                                   
                                                                                
* Number of explicit operands named per instruction.                            
                                                                                
* Location of the operands in the CPU and how                                   
  operand locations are specified.                                              
                                                                                  
  * Type and size of operand                                                      
                                                                                  
  Categories of Instructions                                                      
  1) data transfer                                                                
  2) arithmetic                                                                   
  3) logical                                                                      
  4) control transfer                                                             
  5) I/O                                                                          
  6) system                                                                       
  7) floating point                                                               
  8) decimal                                                                      
  9) string 

AVR has 134 instructions

\line(1,0){150}

Program memory is non-volatile FLASH, data memory is volatile SRAM. Data
memory 0x0000-0x001F is 32 8-bit GPRs (R0-R31); 0x0020-0x005F is 64 8-bit I/O
registers; 0x0060-0x00FF is 160 ext. 8-bit I/O regs; 0x100-0x10FF is 4096 x
8-bit SRAM. Flash is 64Kx16 bits wide.

X is R27(high byte) and R26(low byte), Y R29 and R28, Z R31 and R30.

Status register (SREG) 
Bit 7: Global Interrupt Enable (I)
Bit 6: Bit-copy storage (T)
Bit 5: HalfCarry Flag (H)
Bit 4: Sign bit (S)
Bit 3: Two's complement overflow (V)
Bit 2: Negative flag (N)
Bit 1: Zero Flag (Z)
Bit 0: Carry Flag (C)

\line(1,0){150}

LSL/LSR Logical Shift Left/right, shifts a 0 in (spare bit goes into carry?)

ROL/ROR rotate left or right through carry

SBI/CBI set/clear bit in I/O, SBI/CBI A, bit

SEf/CLf set/clear flag; SEI is turn on global interrupt

\line(1,0){150}

Store Accumulator Indirect with Pre-Decrement

STA -(x) ; M(x) $\leftarrow$ M(x)-1, M(M(x)) $\leftarrow$ AC,

An instruction consists of 16 bits: A 4-bit operation code and a 12-bit
address. All operands are 16 bits. PC and MAR each contain 12 bits. AC, MDR,
and TEMP each contain 16 bits, and IR is 4 bits.


\begin{verbatim}
Fetch Cycle
1: MAR <- PC;
2: MDR <- M(MAR), PC <- PC+1
3: IR <- MDRopc, MAR<- MDRadr 
; Read inst. & increment PC
\end{verbatim}

\begin{verbatim}
Execute Cycle
Step 1: MDR <- M(MAR) 
; Get EA+1 from mem (i.e., M(x))
2: TEMP <- AC ; Save AC to TEMP
3: AC <- MDR ; Decrement EA+1
4: AC <- AC -1
5: MDR <- AC ; Store EA into M(x)
6: M(MAR) <- MDR
7: AC <- TEMP ; Restore AC
8: MAR <- MDR ; point MAR to EA
9: MDR <- AC 
; Store AC into M(M(x))=M(EA)
Step 10: M(MAR) <- MDR
; Note that you can also perform 
; steps 1-2 and 6-7 simultaneously.
\end{verbatim}


\line(1,0){150}

Store Accumulator with Post-Increment
STA (x)+ ; M(M(x)) $\leftarrow$ AC, M(x) $\leftarrow$ M(x)+1

\begin{verbatim}
Fetch: 
1: MAR <- PC
2: MDR <- M(MAR), PC <- PC+1
3: IR <- MDRopc, MAR <- MDRadr

Execute: 
1: MDR <- M(MAR), TEMP <- AC
2: AC <- MDR
3: AC <- AC+1
4: MDR <- AC
5: M(MAR) <- MDR, AC <- AC-1
6: MAR <- AC
7: MDR <- TEMP, AC <- TEMP
8: M(MAR) <- MDR
\end{verbatim}

\line(1,0){150}

Increment and Skip if Zero (ISZ)

ISZ Y ; M(Y) $\leftarrow$ M(Y) + 1, if (M(Y)+1 = 0) then PC $\leftarrow$ PC+1

\begin{verbatim}
Fetch: 
1. MAR <- PC
2. MDR <- M(MAR), PC <- PC+1
3. IR <- MDRopc, MAR <- MDRadr

Execute: 
1. MDR <- M(MAR), TEMP <- AC
2. AC <- MDR
3. AC <- AC+1
4. MDR <- AC
5: M(MAR) <- MDR, AC <- TEMP, 
   if(AC==0)then PC <- PC+1

\end{verbatim}

\line(1,0){150}

ST -X, R3

\begin{verbatim}
Fetch: 
1. MAR <- PC
2. MDR <- M(MAR), PC <- PC+1
3. IR(15...8)<- MDR
4. MAR <- PC
5. MDR <- M(MAR), PC <- PC+1
6. IR(7...0) <- MDR
\end{verbatim}

Next, we need to decrement register X, which is in registers R26 (XH) and R27
(XL). Once X is pre-decremented, the address is put into MAR, content of R27
is put into MDR, and stored into the memory. Note that it is also possible to
perform Steps 4 and 6 and Steps 5 and 7 in parallel. 

\begin{verbatim}
1. AC(15..8) <- R27
2. AC(7..0) <- R26
3. AC <- AC-1
4. R27 <- AC(15..8), 
   MAR(15..8) <- AC(15..8)
5. R26 <- AC(7..0), 
   MAR(7..0) <- AC(7..0)
6. MDR <- R3
7. M(MAR) <- MDR
\end{verbatim}

\line(1,0){150}

RET

\begin{verbatim}
1. MAR <- PC
2. MDR <- M(MAR), PC <- PC+1
3. IR(15..8) <- MDR
4. MAR <- PC
5. MDR <- M(MAR), PC <- PC+1
6. IR(7..0) <- MDR
7. SP <- SP+1
8. MAR <- SP
9. MDR <- M(MAR), SP <- SP+1
10. PC(15..8) <- MDR
11. MAR <- SP
12. MDR <- M(MAR)
13. PC(7..0) <- MDR
\end{verbatim}

\line(1,0){150}

Jump to subroutine: JMS Y ; M(Y) $\leftarrow$ PC, PC $\leftarrow$ Y + 1

\begin{verbatim}
Fetch Cycle
1: MAR <- PC
2: MDR <- M(MAR), PC <- PC+1
3: IR <- MDRopc, MAR <- MDRadr
4: Goto Execute cycle
AND Y ; AC <- AC + M(Y)
Execute Cycle
1: MDR <- M(MAR)
2: AC <- AC + MDR
3: Goto Fetch cycle
\end{verbatim}



\end{multicols}
\end{document}
